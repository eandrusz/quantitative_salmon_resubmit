% Options for packages loaded elsewhere
\PassOptionsToPackage{unicode}{hyperref}
\PassOptionsToPackage{hyphens}{url}
%
\documentclass[
]{article}
\usepackage{amsmath,amssymb}
\usepackage{lmodern}
\usepackage{iftex}
\ifPDFTeX
  \usepackage[T1]{fontenc}
  \usepackage[utf8]{inputenc}
  \usepackage{textcomp} % provide euro and other symbols
\else % if luatex or xetex
  \usepackage{unicode-math}
  \defaultfontfeatures{Scale=MatchLowercase}
  \defaultfontfeatures[\rmfamily]{Ligatures=TeX,Scale=1}
\fi
% Use upquote if available, for straight quotes in verbatim environments
\IfFileExists{upquote.sty}{\usepackage{upquote}}{}
\IfFileExists{microtype.sty}{% use microtype if available
  \usepackage[]{microtype}
  \UseMicrotypeSet[protrusion]{basicmath} % disable protrusion for tt fonts
}{}
\makeatletter
\@ifundefined{KOMAClassName}{% if non-KOMA class
  \IfFileExists{parskip.sty}{%
    \usepackage{parskip}
  }{% else
    \setlength{\parindent}{0pt}
    \setlength{\parskip}{6pt plus 2pt minus 1pt}}
}{% if KOMA class
  \KOMAoptions{parskip=half}}
\makeatother
\usepackage{xcolor}
\usepackage[margin=1in]{geometry}
\usepackage{graphicx}
\makeatletter
\def\maxwidth{\ifdim\Gin@nat@width>\linewidth\linewidth\else\Gin@nat@width\fi}
\def\maxheight{\ifdim\Gin@nat@height>\textheight\textheight\else\Gin@nat@height\fi}
\makeatother
% Scale images if necessary, so that they will not overflow the page
% margins by default, and it is still possible to overwrite the defaults
% using explicit options in \includegraphics[width, height, ...]{}
\setkeys{Gin}{width=\maxwidth,height=\maxheight,keepaspectratio}
% Set default figure placement to htbp
\makeatletter
\def\fps@figure{htbp}
\makeatother
\setlength{\emergencystretch}{3em} % prevent overfull lines
\providecommand{\tightlist}{%
  \setlength{\itemsep}{0pt}\setlength{\parskip}{0pt}}
\setcounter{secnumdepth}{-\maxdimen} % remove section numbering
\newlength{\cslhangindent}
\setlength{\cslhangindent}{1.5em}
\newlength{\csllabelwidth}
\setlength{\csllabelwidth}{3em}
\newlength{\cslentryspacingunit} % times entry-spacing
\setlength{\cslentryspacingunit}{\parskip}
\newenvironment{CSLReferences}[2] % #1 hanging-ident, #2 entry spacing
 {% don't indent paragraphs
  \setlength{\parindent}{0pt}
  % turn on hanging indent if param 1 is 1
  \ifodd #1
  \let\oldpar\par
  \def\par{\hangindent=\cslhangindent\oldpar}
  \fi
  % set entry spacing
  \setlength{\parskip}{#2\cslentryspacingunit}
 }%
 {}
\usepackage{calc}
\newcommand{\CSLBlock}[1]{#1\hfill\break}
\newcommand{\CSLLeftMargin}[1]{\parbox[t]{\csllabelwidth}{#1}}
\newcommand{\CSLRightInline}[1]{\parbox[t]{\linewidth - \csllabelwidth}{#1}\break}
\newcommand{\CSLIndent}[1]{\hspace{\cslhangindent}#1}
\usepackage{lineno}
\linenumbers
\usepackage{setspace}\doublespacing
\usepackage{gensymb}
\geometry{verbose,letterpaper,tmargin=2.54cm,bmargin=2.54cm,lmargin=2.54cm,rmargin=2.54cm}
\usepackage{float}
\ifLuaTeX
  \usepackage{selnolig}  % disable illegal ligatures
\fi
\IfFileExists{bookmark.sty}{\usepackage{bookmark}}{\usepackage{hyperref}}
\IfFileExists{xurl.sty}{\usepackage{xurl}}{} % add URL line breaks if available
\urlstyle{same} % disable monospaced font for URLs
\hypersetup{
  hidelinks,
  pdfcreator={LaTeX via pandoc}}

\author{}
\date{\vspace{-2.5em}}

\begin{document}

\hypertarget{quantifying-impacts-of-an-environmental-intervention-using-environmental-dna}{%
\subsection{Quantifying Impacts of an Environmental Intervention Using
Environmental
DNA}\label{quantifying-impacts-of-an-environmental-intervention-using-environmental-dna}}

Elizabeth Andruszkiewicz Allan\(^{1*\ddagger}\), Ryan P. Kelly\(^{1*}\),
Erin R. D'Agnese\(^{1,3}\), Maya N. Garber-Yonts\(^{1}\), Megan R.
Shaffer\(^{1}\), Zachary J. Gold\(^{2\dagger}\), Andrew O.
Shelton\(^{2}\)

\(^{1}\) University of Washington, School of Marine and Environmental
Affairs, 3737 Brooklyn Ave NE, Seattle, WA 98105, U.S.A.

\(^2\) Conservation Biology Division, Northwest Fisheries Science
Center, National Marine Fisheries Service, National Oceanic and
Atmospheric Administration, 2725 Montlake Blvd. E, Seattle, WA 98112,
U.S.A.

\(^3\) Wild EcoHealth, Tacoma WA, 98465, USA

\vspace{1em}

\(^{*}\) Authors contributed equally to this work.

\(^{\dagger}\) Currently at NOAA Pacific Marine Environmental
Laboratory, Seattle, WA, USA.

\(^{\ddagger}\) Corresponding author:
\href{mailto:eallan@uw.edu}{\nolinkurl{eallan@uw.edu}} \vspace{1em}

For submission to: \textit{Ecological Applications} \newline Manuscript
type: Article \newline Open Research Statement: Data are already
published and publicly available, with those publications properly cited
in this submission. The repository for code can be found at:
\newline Andruszkiewicz Allan et al.~2023.
\url{https://doi.org/10.5281/zenodo.8029271}
\newline \textit{Keywords: environmental DNA, quantitative metabarcoding, environmental impact assessments, salmon, culvert}

\newpage

\hypertarget{abstract}{%
\subsection{Abstract}\label{abstract}}

Environmental laws around the world require some version of an
environmental impact assessment surrounding construction projects and
other discrete instances of human development. Information requirements
for these assessments vary by jurisdiction, but nearly all require an
analysis of biological elements of ecosystems. Amplicon-sequencing -
also called metabarcoding - of environmental DNA (eDNA) has made it
possible to sample and amplify the genetic material of many species
present in those environments, providing a tractable, powerful, and
increasingly common way of doing environmental impact analysis for
development projects. Here, we analyze a 18-month time-series of water
samples taken before, during, and after two culvert removals in a
salmonid-bearing freshwater stream. We also sampled multiple control
streams to develop a robust background expectation against which to
evaluate the impact of this discrete environmental intervention in the
treatment stream. We generate calibrated, quantitative metabarcoding
data from amplifying the 12s MiFish mtDNA locus and complementary
species-specific quantitative PCR data to yield multi-species estimates
of absolute eDNA concentrations across time, creeks, and sampling
stations. We then use a linear mixed-effects model to reveal patterns of
eDNA concentrations over time, and to estimate the effects of the
culvert removal on salmonids in the treatment creek. We focus our
analysis on four common salmonid species: cutthroat trout
(\emph{Oncorhynchus clarkii}), coho salmon (\emph{O. kisutch}), rainbow
trout (\emph{O. mykiss}), and sockeye salmon (\emph{O. nerka}). We find
that one culvert in the treatment creek seemed to have no impact while
the second culvert had a large impact on fish passage. The construction
itself seemed to have only transient effects on salmonid species during
the two construction events. In the context of billions of dollars of
court-mandated road culvert replacements taking place in Washington
State, USA, our results suggest that culvert replacement can be
conducted with only minimal impact of construction to key species of
management concern. Furthermore, eDNA methods can be an effective and
efficient approach for monitoring hundreds of culverts to prioritize
culverts that are required to be replaced. More broadly, we demonstrate
a rigorous, quantitative method for environmental impact reporting using
eDNA that is widely applicable in environments worldwide.

\newpage

\hypertarget{introduction}{%
\subsection{Introduction}\label{introduction}}

At present, it remains difficult to comprehensively measure the
environmental impacts of discrete human activities, despite such
assessment often being required by law. Within the United States, both
state and federal laws often require a form of environmental-impact
assessment for medium- to large-scale projects (i.e., those that might
have a significant impact on the environment) (Morgan 2012). Outside the
US, many nations have their own versions of these same laws.
Specifically when measuring impacts on aquatic ecosystems, assessments
generally are based on literature reviews or field measurements of key
species selected beforehand (Rubin et al. 2017). These traditional
methods are often expensive, rely on just a few species, and are limited
in spatial and temporal coverage (Martin et al. 2012). Moreover, they
often lack pre-project monitoring and any or sufficient post-project
monitoring, given that the goals of a development project normally focus
on construction itself and funding is often extremely limited. For
example, a recent literature review of stream restoration projects cited
that more than half of projects evaluated (62\%) had no pre-project
monitoring and only sampled once per year (for before, during, and
post-project sampling) (Rubin et al. 2017). Thus, current assessment
efforts relying on traditional survey methods often fall short in
documenting and quantifying environmental impacts.

A key difficulty in conducting ecosystem assessments is that there is no
one way to survey the world and just ``see what is there.'' All methods
of environmental sampling are biased as they capture a selective portion
of the biodiversity present (Rubin et al. 2017). Net samples for fish,
for example, fail to capture species too small or too large to be caught
in the net. Environmental DNA (eDNA), however, comes as close to this
goal as any method yet developed although not without bias (see below):
a sample of water, soil, or even air, contains the genetic traces of
many thousands of species, from microbes to whales. Sequencing eDNA is
therefore a means of surveying many species in a consistent and scalable
way (Taberlet et al. 2012, Thomsen and Willerslev 2015). Environmental
assessments have begun to make use of eDNA for such work around the
world (Muha et al. 2017, Duda et al. 2021, Klein et al. 2022, Maasri et
al. 2022, Moss et al. 2022), but are not yet common practice. Surveying
the world by eDNA has long been commonplace in microbial ecology (Ogram
et al. 1987, Rondon et al. 2000, Turnbaugh et al. 2007) but has recently
become popular for characterizing eukaryotic communities (Taberlet et
al. 2012, Kelly et al. 2014, De Vargas et al. 2015, Port et al. 2015,
Valentini et al. 2016, Stat et al. 2017). Techniques generally include
an amplification step such as quantitative PCR, digital or
digital-droplet PCR, or traditional PCR from mixed templates followed by
high-throughput sequencing (Ruppert et al. 2019). This last technique is
known as eDNA metabarcoding.

In a metabarcoding approach, broad-spectrum PCR primers identify
hundreds or thousands of taxa across a very wide diversity of the tree
of the life (e.g., Leray et al. (2013)). Nevertheless the absence of a
taxon from a sequenced sample does not indicate the absence of that
taxon from the environment but rather that the taxon failed to amplify
(Shelton et al. 2016, Kelly et al. 2019, Buxton et al. 2021, Gold et al.
2023). In virtually all comparisons, metabarcoding recovers far more
taxa than any other sampling method (Port et al. 2015, Kelly et al.
2017, Seymour et al. 2021). However, we expect results from
metabarcoding to differ dramatically from non-PCR based sampling methods
due to the fundamental differences in sampling residual genetic material
as opposed to whole organisms. Furthermore, eDNA analyses rely on
several laboratory processes, including PCR amplification, all of which
contribute to complicating the interpretation of results (see Shelton et
al. (2016) and Kelly et al. (2019)). Specifically, PCR amplification is
an exponential process for which the efficiency varies across species
and primer set (Gloor et al. 2016). By understanding these differences,
we can correct for taxon-specific biases to yield quantitative estimates
of the community composition prior to PCR (McLaren et al. 2019, Shelton
et al. 2022). Other ways to conduct quantitative metabarcoding include
using qSeq (Hoshino et al. 2021), a process in which a random tag is
added to target sequences before PCR. However, if different species
amplify at different rates during PCR, these quantifications would
reflect not just the starting concentration but also the amplification
efficiency.

After correcting for amplification biases, the resulting dataset is
compositional, revealing the proportions of each species' DNA present in
each sample, but importantly, contains no information about the absolute
abundance of DNA present (Gloor et al. 2016, McLaren et al. 2019,
Silverman et al. 2021, Shelton et al. 2022). We can tie these
proportional estimates to absolute abundances using additional data such
as a quantitative PCR (qPCR) assay for one of the taxa present. Thus, a
single qPCR assay and a single metabarcoding assay can together provide
quantitative estimates of many species as opposed to running as many
qPCR assays as species of interest (see also (Pont et al. 2022)).
Together, we can use these data to assess changes in eDNA concentrations
of species over time, and due to environmental impacts, such as
replacing a culvert under a road.

As a result of a ruling in a federal court (Martinez 2013), Washington
State is under a mandate to replace hundreds of culverts that allow
water to pass under roads and highways, costing billions of dollars.
Improperly designed culverts can lead to many negative consequences for
fish, especially anadromous salmon, including habitat fragmentation,
loss of accessibility to spawning and rearing habitat, and genetic
isolation (Price et al. 2010, Frankiewicz et al. 2021). These impacts on
salmonids furthermore violate the sovereign treaty rights of the
region's indigenous tribes (Martinez 2013). Salmonid species are of
cultural and economic importance to the indigenous peoples of the
region, and without restoration of historic salmon-rearing habitat, the
continued decline of salmonids can lead to not only ecological
destruction, but the loss of cultural and economic viability for many
indigenous tribes (Schmidhauser 1976, Lackey 2003, Long and Lake 2018).

Presently, the prioritization process for replacing culverts preventing
fish passage conducted by the Washington Department of Transportation is
a protocol provided by the Washington Department of Fish and Wildlife,
which includes factors such as the amount of habitat blocked by the
barrier, the types of species blocked by the barrier, and estimated cost
of repair, among other things (Washington Department of Fish and
Wildlife 2019). However, data on fish presence upstream of barriers
(i.e., culverts blocking fish passage) are rare and often not included
in these assessments. Using eDNA as a proxy for fish presence could
provide important data for project prioritization and have the potential
to be more cost effective.

Once a culvert has been designated as in need of repair, the intention
is to improve conditions for biota, including migrating fish, but the
construction itself might have a short-term negative effect before the
longer-term improvements are realized. Specifically in culvert
replacements, studies have cited the negative impacts of construction to
include sediment accumulation, removal of vegetation, and blocking flow
and stranding fish (Wellman et al. 2000, Washington Department of Fish
and Wildlife 2019). However, it is unclear how long these effects might
last and if the long-term benefits of the culvert replacement justify
the short-term costs of the construction. These disruptions also
underscore the importance of both properly assessing culverts to
determine if they are blocking fish passage and monitoring after
construction to ensure the replacement actually improved fish passage.

Many studies have attempted to quantify when culverts are barriers to
fish passage and how effective culvert replacements are for fish
passage, either by measuring physical parameters of the culvert and
stream after replacement (Price et al. 2010), or by measuring biological
parameters, including electrofishing (Ogren and Huckins 2015) or genetic
differentiation from fish tissues (Wood et al. 2018, Nathan et al.
2018). In some cases, culverts deemed blockages did not prove to block
fish passage (MacPherson et al. 2012), while in others, blockages that
were replaced were not found to improve fish passage (Price et al. 2010)
or improve overal biotic integrity (Ogren and Huckins 2015). Sampling
water for eDNA analysis before, during, and post-restoration can provide
valuable information on if the restoration is needed, how the
restoration negatively impacts communities during construction, and if
the restoration efforts did in fact correct the blockage.

Here, we report the results of an approximately 18-month eDNA sampling
effort before, during, and after the replacement of two culverts (one
small and one large) in a creek, assessing the impact of these projects
on the salmonid species present. We do so using a combination of
metabarcoding (12s mtDNA) and qPCR to yield estimates of the
concentrations of DNA present at each time point, and we use parallel
samples from four control creeks to develop a causal analysis of changes
in these concentrations. A clear opportunity for policy-relevant eDNA
work is in using its power to survey many species at a time to improve
the way we assess the impacts of human activities. Here, we demonstrate
the utility of eDNA for policy-relevant environmental assessments by
surveying many species simultaneously and improving the way we assess
the impacts of human activities.

\hypertarget{methods}{%
\subsection{Methods}\label{methods}}

\hypertarget{site-and-species-selection}{%
\subsubsection{Site and Species
Selection}\label{site-and-species-selection}}

We selected sampling locations based loosely on an asymmetrical BACI
(Before-After-Control-Impact) study design (Underwood 1992, 1994,
Benedetti-Cecchi 2001) to measure the environmental impact of a culvert
replacement using eDNA. We sampled four control creeks where
construction was not occurring (Figure 1) at monthly intervals, both
upstream and downstream of each creek's culvert. The two culverts in the
treatment creek (Padden) were suspected to be partially impassible and
thus were removed and replaced during the course of the study. The four
control creeks ranged from preventing fish passage (Barnes and
Chuckanut), partially passable (Squalicum), to allowing fish passage
(Portage; see Appendix S1) (Washington Department of Fish and Wildlife
2019). These creeks were chosen due to their comparable size, flow,
watersheds, and species presumed to be present to constrain as many
ecological variables as possible.

The first culvert replacement (SR-11) in Padden Creek occurred over
about two months and included the ``de-watering'' of the creek, removal
of the existing culvert, installation of the new culvert, and then the
``re-watering'' of the creek from late August 2021 to early October 2021
(Appendix S1: Figure S4). The second culvert replacement (I-5) in Padden
Creek was a much larger construction project, including daylighting the
creek and building a bridge under a large, five-lane interstate.
In-water work for the I-5 culvert replacement began in late June 2022
and was completed in September 2022. By sampling before, during, and
after both construction events, we were then able to isolate the effect
of the culvert replacement itself -- controlling for temporal trends,
background environmental variability, and sampling variability -- using
a linear mixed effects model of eDNA abundances across creeks, time
points, sampling stations, and species.

Because salmonids are the primary species of management concern in these
creeks, we focus the present analysis on the four salmonid species most
common in our data: cutthroat trout (\emph{Oncorhynchus clarkii)}, coho
salmon (\emph{O. kisutch}), rainbow/steelhead trout (\emph{O. mykiss}),
and sockeye/kokanee salmon (\emph{O. nerka}). Not all four salmonids are
expected to be found in all five of the creeks sampled. As documented by
WA Department of Fish and Wildlife SalmonScape
(\url{http://apps.wdfw.wa.gov/salmonscape/map.html}), all creeks contain
cutthroat trout, steelhead trout, and coho salmon. Barnes Creek is the
only creek documented to have kokanee salmon (a freshwater sub-type of
sockeye salmon). However, local spawner surveys conducted by the City of
Bellingham from 2015-2020 in Padden Creek documented kokanee salmon, as
well as the other three species (City of Bellingham 2015). The four
salmonid species in this study have different life histories and
behaviors that would impact when fish (and therefore eDNA
concentrations) occur in the creeks. Furthermore, three of the four
species in this study have both freshwater resident and saltwater
migrating behavior. For the fish exhibiting migratory behavior, the run
timings vary for each species in the study area (see Discussion and
Appendix S1: Figure S4). Therefore, our eDNA concentrations might
reflect contributions from both migrating and non-migrating individuals
at any given time point in the dataset.

\hypertarget{water-sampling}{%
\subsubsection{Water Sampling}\label{water-sampling}}

From March 2021 to February 2022, all five creeks were sampled monthly
(n=12). Monthly sampling continued in Portage Creek, Padden Creek, and
Squalicum Creek through August 2022, with one additional sampling point
in December 2022 (n=19). At each sampling station (upstream and
downstream of a culvert) at each creek, we collected three 2-liter water
samples. Samples were collected using an eDNA Backpack (Smith Root;
Thomas et al. (2018)), a portable pumping-and-filtering device set to
filter at 1 L/min at 82.7 kPa (12 psi). In some months, less than 2 L of
water was filtered due to clogging. Water samples were filtered using
single-use inlet tubes through 5\(\mu\)m self-preserving filters (Smith
Root, Vancouver, WA), which were then dried and kept at room temperature
until DNA extraction within 1 month of collection (Thomas et al. 2019).

Over the course of the sampling, water discharge varied from very low to
no flow in summer months to high flow in winter months (Figure 2). We
account for this dilution by converting eDNA concentration
{[}copies/\(\mu\)L{]} to an eDNA mass flow rate {[}copies/s{]} by
multiplying eDNA concentrations by discharge {[}L/s{]} (Tillotson et al.
2018, Thalinger et al. 2019). Flow gauges maintained by the United
States Geological Survey (USGS) were used for Padden Creek (USGS Gauge
12201905), Chuckanut Creek (USGS Gauge 12201700), and Squalicum Creek
(USGS Gauge 12204010;
\url{https://maps.waterdata.usgs.gov/mapper/index.html}; U. S.
Geological Survey (1994); Appendix S1: Figure S1). Over the course of
sampling, the flow gauges at Chuckanut Creek and Squalicum Creek became
inoperable after a major flooding event. To find discharge rates for
Chuckanut and Squalicum Creeks, five years of historical data
(2015-2020) were used to generate a monthly averaged correction factor
based on Padden Creek (Appendix S1, Appendix S1: Figure S3). No
discharge data was available for Portage Creek or Barnes Creek. Based on
field sampling conditions, the discharge from Padden Creek was used as a
proxy for both Portage and Barnes as they are in similarly sized
watershed areas and land-cover characteristics.

\hypertarget{dna-extraction-amplification-sequencing}{%
\subsubsection{DNA Extraction, Amplification,
Sequencing}\label{dna-extraction-amplification-sequencing}}

All molecular work prior to sequencing was performed at the University
of Washington. Details of the molecular work can be found in Appendix
S1. Briefly, DNA was extracted off filters using a Qiashredder column
(Qiagen, USA) and the DNeasy Blood and Tissue Kit (Qiagen, USA) with an
overnight incubation (Appendix S1, Thomas et al. (2019)). Extracts were
stored at -20\degree C until PCR amplification within 2 months of
extraction.

For metabarcoding, we targeted a \textasciitilde186 bp hypervariable
region of the mitochondrial DNA 12S rRNA gene for PCR amplification
(MiFish; Miya et al.~2015), but using modified primer sequences as given
in Praebel and Wangensteen (unpublished; via personal communication).
The primer sequences, final reaction recipe, and cycling conditions can
be found in Appendix S1. Each month of samples was amplified on a single
plate with the addition of a no template control (NTC; molecular grade
water in lieu of template) and a positive control (genomic DNA from
kangaroo, a species not present in the environment). PCR products were
visualized, size-selected, and diluted iteratively if inhibited. After
cleaning, a second PCR amplification added unique indices to each sample
using Nextera indices (Illumina, USA) to allow pooling multiple samples
onto the same sequencing run (See Appendix S1 for details). Indexed PCR
products were also size-selected and visualized before pooling for
sequencing. Samples were randomized in 3-month blocks and each block
split across 3 sequencing runs to avoid run effects, for a total of 14
sequencing runs. The loading concentration of each library was 4-8 pM
and 5-20\% PhiX was included depending on the composition of the run.
Sequencing was conducted using an Illumina Miseq with v3 2x300 chemistry
at the NOAA Northwest Fisheries Science Center and the University of
Washington's Northwest Genomics Center.

Here, we used mock communities to determine the species-specific
amplification efficiencies for each salmonid in the study. We
constructed five communities with known proportions of starting DNA from
different species (total DNA as measured by Qubit). The communities
ranged from having a total of 12 to 20 species, but six salmonid species
were included in all five mock communities (Appendix S3: Table S2). We
sequenced these communities using the same metabarcoding primers and
thermocycling conditions above and then determined the species-specific
amplification rates given the discrepancy between the known starting
proportion and the proportion of reads after sequencing. The mock
community data were then used to correct the sequencing reads from the
environmental samples to estimate the starting DNA proportions of each
species in environmental samples, which is the metric of interest
(Figure 3, green boxes). This is the first application of the model to
correct eDNA data from water samples with mock community data as
described in Shelton et al. (2022) (see Appendix S2 for more
information).

\hypertarget{bioinformatics}{%
\subsubsection{Bioinformatics}\label{bioinformatics}}

After sequencing, bioinformatic analyses were conducted in R (R Core
Team 2017). A more detailed description of the bioinformatics pipeline
is included in Appendix S1. Briefly, primer sequences were removed using
\emph{Cutadapt} (Version 1.18) (Martin 2011) before \emph{dada2}
(Callahan et al. 2016) trimmed, filtered, merged paired end reads, and
generated amplicon sequence variants (ASVs). Taxonomic assignment was
conducted via the \emph{insect} package (Wilkinson et al. 2018) using a
tree generated by the developers for the MiFish primers that was last
updated in November 2018. Only species level assignments from
\emph{insect} were retained and ASVs not annotated or not annotated to
species level were then checked against the NCBI nucleotide database
using BLAST+ (Camacho et al. 2009). Query sequences that matched a
single species at \textgreater95\% identity were retained.

\hypertarget{quantitative-pcr-and-inhibition-testing}{%
\subsubsection{Quantitative PCR and Inhibition
Testing}\label{quantitative-pcr-and-inhibition-testing}}

We quantified cutthroat trout (\emph{O. clarkii}) DNA in each sample,
targeting a 114 bp fragment of the cytochrome b gene with a qPCR assay
(Duda et al. 2021). The primer/probe sequences, final recipe, and
thermocycling conditions can be found in Appendix S1. Each DNA sample
was run in triplicate and was checked for inhibition using the EXO-IPC
assay (Applied Biosystems). The majority of environmental samples (60\%)
were inhibited and accordingly diluted for analysis. In 80\% of
inhibited samples, a 1:10 dilution or less remedied the inhibition, but
some samples (11\%) required dilution by a factor of up to 1000. Each
plate included a 8-point standard curve created using synthetic DNA
(gBlocks) ranging from 1 to 100,000 copies/\(\mu\)L and six no template
controls (NTCs) were included on each plate with molecular grade water
instead of template. All qPCRs were conducted on an Applied Biosystems
StepOnePlus thermocycler.

All qPCR data was processed in R using Stan (Stan Development Team
2022), relating environmental samples to the standard curve via a linear
model (Figure 3, blue boxes; Appendix S2: Figure S1). We amended the
standard linear regression model to more realistically capture the
behavior of qPCR observations, accommodating non-detections as a
function of underlying DNA concentration, and letting the standard
deviation vary with the mean (lower-concentration samples had more
uncertainty). See McCall et al. (2014) and Shelton et al. (2019) for
similar models; see Appendix S2 for full statistical details. Subsequent
analysis corrected for sample-specific dilution if found inhibited and
corrected for any variation in water-volume filtered during sample
collection. Samples with standard deviations between technical
replicates larger than 1.5 Ct values were removed from analyses.

\hypertarget{quantitative-metabarcoding}{%
\subsubsection{Quantitative
Metabarcoding}\label{quantitative-metabarcoding}}

The intercalibration of the mock community samples demonstrated the rank
order of amplification efficiencies for salmonids (Appendix S1: Figure
S10 and Appendix S1: Figure S11). Cutthroat trout (\emph{O. clarkii})
and sockeye/kokanee salmon (\emph{O. nerka}) had similar amplification
efficiencies, both of which were higher than rainbow/steelhead trout
(\emph{O. mykiss}) and coho salmon (\emph{O. kisutch}), which had the
lowest amplification efficiency. Calibrated metabarcoding analysis
yielded quantitative estimates of the proportions of species' DNA in
environmental samples prior to PCR. We then converted these proportions
into absolute abundances by expansion, using the qPCR results for our
reference species, cutthroat trout (\emph{O. clarkii}). We estimated the
total amplifiable salmonid DNA in environmental sample \(i\) as
\(\text{C}_{\text{amplifiable}_{i}} = \frac{\text{C}_{\text{qPCR reference}_{i}}}{\text{Proportion}_{\text{reference}_{i}}}\),
where C has units of {[}DNA copies/uL{]} and then expanded species'
proportions into absolute concentrations by multiplying these
sample-specific total concentrations by individual species' proportions,
such that for species \(j\) in sample \(i\),
\(\text{C}_{i,j} = \text{C}_{\text{amplifiable}_{i}} * \text{Proportion}_{i,j}\).
Here, we combine the modeled output of the qPCR model for cutthroat
trout (Figure 3, dashed blue box) and modeled proportions of salmonid
DNA from metabarcoding (Figure 3, dashed green box). Although in the
future this could be used as a joint model, here the precision of our
modeled estimates were very high such that we used the mean of the
posterior estimates from each model to move forward as input to the time
series model (Figure 3, dashed purple box; see Appendix S2 for more
details). Finally, due to the range of water discharge over the course
of the year, we converted from DNA concentration {[}copies/L{]} to a
mass flow rate {[}copies/s{]} after multiplying by the discharge of each
creek {[}m\textsuperscript{3}/s{]} (Figure 3, solid purple boxes).

\hypertarget{estimating-the-effects-of-culvert-replacement-and-of-culverts-themselves}{%
\subsubsection{Estimating the Effects of Culvert Replacement and of
Culverts
Themselves}\label{estimating-the-effects-of-culvert-replacement-and-of-culverts-themselves}}

We sampled four control creeks as context against which to compare the
observations in Padden Creek, our treatment creek where the two culverts
were being replaced. At a given station in a given creek, some DNA
concentration exists for each species. For simplicity, we focus on a
single species and a single station (downstream or upstream) for the
moment. Our observations of the (log) DNA concentration in creek \(i\)
at time \(t\) are distributed as
\(Y_{i,t} \sim \mathcal{N}(\mu_{i,t},\,\sigma^{2})\). More complex
versions of the model may let \(\sigma\) vary across creeks, time
points, species, or with environmental covariates of interest.

We are interested in how the DNA concentration changes over time, so we
assert that the expected value of DNA in a creek at time \(t\),
\(\mu_{i,t}\), depends upon its time point, in some way. We considered
three ways of modeling the salmonid eDNA data, each in a Bayesian
framework, but each treating non-independence among time points somewhat
differently:

\begin{itemize}
\item
  A linear auto-regressive (AR(1)) model, written in~\texttt{stan}. For
  each species in each creek, the expected concentration of eDNA of each
  month is a linear function of the expected value from the previous
  month. Within a species, the monthly autoregressive parameters are
  shared across creeks.
\item
  A generalized additive model (GAM), written in~\texttt{brms}~(which
  itself writes a~\texttt{stan}~model). For each species in each creek,
  an independent set of spline (weighting) parameters describes the
  temporal trends in expected eDNA concentration; the number of spline
  knots is shared across species and creeks.
\item
  A linear mixed-effects (LME) model, written in~\texttt{rstanarm}. For
  each species in each creek, sampling month is treated as a random
  effect. Each species-creek-month effect is treated as an independent
  draw from a common distribution.~
\end{itemize}

Ultimately, the three models yielded very similar results (Appendix S2),
and the LME model proved simplest and most flexible insofar as it could
easily handle datasets with uneven sets of observations -- for example,
cases in which a species was detected downstream of a barrier, but not
upstream.

In \texttt{R} code using \texttt{rstanarm}, this model is coded as

\begin{verbatim}
stan_glmer(log(observed) ~ (1 + time_idx|creek:species) + (1|station:creek:species:time_idx)  
\end{verbatim}

\hypertarget{results}{%
\subsection{Results}\label{results}}

\hypertarget{metabarcoding-and-quantitative-pcr}{%
\subsubsection{Metabarcoding and Quantitative
PCR}\label{metabarcoding-and-quantitative-pcr}}

In total, we generated \textasciitilde52 million reads across all
environmental samples and 27 mock community samples (3 communities x 9
replicates {[}6 even, 3 skewed proportions{]}) for calibration (see
below). After quality-filtering and merging all runs, \textasciitilde45
million reads remained from \textasciitilde24,000 amplicon sequence
variants (ASVs) in the environmental samples, of which
\textasciitilde83\% of reads were annotated to species level (per
sample: mean = 82\%, median = 93\%, min = 0\%, max = 99.99\% of reads
annotated). We only focus on the metabarcoding data from four salmonids
for the remainder of this paper. The four salmonids represent
\textasciitilde54\% of all reads and \textasciitilde64\% of the
annotated reads found in environmental samples.

In the mock community samples, 98.7\% of the \textasciitilde5 million
reads after quality filtering were annotated to species level.
Importantly, the target salmonid ASVs in the mock communities were found
in environmental samples, unambiguously linking the taxa in calibration
samples with those in environmental samples. The most common salmonid
species found in the environmental samples was cutthroat trout (\emph{O.
clarkii}), which was found in \textasciitilde85\% of samples, followed
by coho salmon (\emph{O. kisutch}) found in \textasciitilde62\% of
samples, then rainbow/steelhead trout (\emph{O. mykiss}) found in
\textasciitilde40\% of samples, and finally sockeye/kokanee salmon
(\emph{O. nerka}) found in \textasciitilde10\% of samples. Over half of
the samples across all times, creeks, and stations had at least 50\% of
reads assigned to cutthroat trout.

After calibrating metabarcoding data using mock communities (See
Appendix S1 and Appendix S2), we estimated the salmonid composition
across time points, creeks, and stations (Figures 4 and 5). The culvert
in one control creek (Barnes) appeared to be nearly a total barrier to
salmonid passage, with salmonid eDNA detected upstream of the culvert at
only three time points, in contrast to being detected at every time
point in the downstream station of the same creek. The other four creeks
had no such pattern associated with the culverts, suggesting that fish
passage may have been possible through the culverts, or that there were
resident populations upstream of the culverts.

All environmental samples were quantified for absolute concentrations of
cutthroat trout DNA across 32 qPCR plates, resulting in
\textasciitilde630 samples (\textasciitilde60\%) with a positive
detection in at least 1 of 3 technical replicates. The modeled output of
cutthroat trout DNA concentrations, ranged from 50 copies/L to 1.4 x
10\textsuperscript{6} copies/L, with a mean value of
\textasciitilde47,000 copies/L (Figure 6).

We combined compositional information from metabarcoding with absolute
concentrations from qPCR for our reference species, cutthroat trout
(\emph{O. clarkii}), to estimate the total concentration of DNA for each
species (See Appendix S2). These quantitative data for all four target
species were then used in the linear mixed-effects model to assess
salmonid trends over time (Figure 7).

\hypertarget{effects-of-culverts}{%
\subsubsection{Effects of Culverts}\label{effects-of-culverts}}

Before considering the effect of construction, the difference in trends
between upstream and downstream stations (Figure 7) demonstrates that
the culverts themselves have some effect, but generally not a large
effect on the salmonid species surveyed. A notable exception was Barnes
Creek, as the culvert was so clearly a barrier as most time points had
no salmonid DNA upstream. Padden Creek upstream of I-5 also was more
clearly a barrier to fish passage, while the culvert across SR-11 seemed
to be less of a barrier to fish passage. In other cases, salmonid DNA is
found upstream but not downstream, indicating that the culvert is likely
not a barrier and there are resident individuals upstream of the
culvert.

Summarizing over all species and the four creeks used in the time series
model, the culvert effect was minimal (Figure 8); the average log-fold
change between upstream and downstream sites was not significantly
different from zero. Individual species' patterns were similar,
indicating that there is not a species-specific effect where culverts
block the passage of some salmon but not others (Appendix S1: Figure
S12). The maximum positive log-fold change (i.e., upstream having a
higher mass flow rate) was 2.78 in Squalicum Creek for coho salmon
(\emph{O. kisutch}) in August 2021, while the maximum negative log-fold
change (i.e., downstream having a higher mass flow rate) was -1.11 found
in Squalicum Creek for cutthroat trout (\emph{O. clarkii}) in December
2021 (Appendix S1: Figure S12). Of all species, creeks, and time points,
23 of the 151 observations were within a log-fold change of -0.1 to 0.1,
which corresponds with eDNA mass flow rates upstream within 10\% of mass
flow rates downstream.

We also considered how the log-fold change in eDNA mass flow rates were
impacted by the flow rate or discharge of the creek itself (Figure 9).
We found that at months of the lowest flow (summer months), the log-fold
changes between mass flow rates were the highest, while in winter months
with highest discharge the log-fold changes were lower and downstream
sites often had higher eDNA mass flow rates than upstream sites
(Appendix S1: Figure S12).

\hypertarget{effects-of-culvert-replacement}{%
\subsubsection{Effects of Culvert
Replacement}\label{effects-of-culvert-replacement}}

By comparing the difference in upstream and downstream mass flow rates
before and after construction in Padden Creek, we can assess how large
of an impact the two culvert replacements had on salmonid species
(Figure 10). The effects of the culvert replacement operations appeared
to have been transient and fairly minor for the four salmonid species
surveyed. We saw very minor fluctuations in the difference between
upstream and downstream salmonid DNA mass flow rates, and did not see an
increase in this difference due to the culvert removal as the log-fold
changes in Padden Creek were similar to those in the control creeks at
the same time points (Figure 10, grey points vs.~black points in areas
of yellow shading).

\hypertarget{discussion}{%
\subsection{Discussion}\label{discussion}}

\hypertarget{environmental-dna-can-provide-quantitative-measurements-of-environmental-impacts}{%
\subsubsection{Environmental DNA can provide quantitative measurements
of environmental
impacts}\label{environmental-dna-can-provide-quantitative-measurements-of-environmental-impacts}}

Here, we used both eDNA metabarcoding and a single species-specific qPCR
assay to rigorously quantify both the effect of culverts and the impact
of two culvert replacements on salmonids in the same creek. We observed
a clear seasonal pattern in the DNA concentrations of four salmonid
species detected in the study. The sampling design and the linear mixed
effects model leveraged information across treatment and control creeks
to integrate the change in eDNA concentrations due to time, whether a
sample was collected below or above a barrier (i.e., culvert), and
whether or not there was construction occurring. Thus, we could isolate
the changes in eDNA concentrations as a result of the intervention
(i.e., construction) while accounting for the variance due to time and
station (i.e., season and culvert).

A few other studies have used eDNA to measure environmental impacts in
rivers and streams. Duda et al. (2021) used 11 species-specific qPCR
assays to document the distribution of resident and migratory fish after
a large dam removal project (Elwha River near Port Angeles, Washington).
No eDNA sampling was conducted before the dam removal, but the study
provided a wealth of information about species returning after the dam
removal, providing a very important dataset to use eDNA to monitor
ecological changes due to human intervention. Similarly, Muha et al.
(2017) sampled three locations upstream and three locations downstream
before and after the removal of a weir that was thought to be a barrier
to salmonid migrations. The authors only sampled once before and twice
after the removal, spanning about a year, and used eDNA metabarcoding to
look at the presence/absence of species detected. They found that in
fact the before sample demonstrated that the weir was not preventing
fish passage (similar to the results found in this study) and
furthermore documented a slight increase in alpha diversity in the first
time point after the barrier removal and then a return to a similar
alpha diversity in the second time point after the removal (similar
results found in this study using eDNA concentrations rather than
diversity). Finally Yamanaka and Minamoto (2016) sampled along a river
with three barriers, finding some fish able to cross barriers and some
not, suggesting that the eDNA can indicated habitat connectivity for
fishes across barriers.

Importantly, our study demonstrates the value of combining a single qPCR
assay with metabarcoding data to generate quantitative estimates of eDNA
concentrations of many species without requiring \emph{n} qPCR assays
for \emph{n} species of interest. Here, we ultimately only quantified
the impacts of four species, but importantly, we did not know \emph{a
priori} how many species of interest there might be and we reduced our
efforts two fold by only conducting two assays (one species-specific
qPCR and one metabarcoding assay) as opposed to four assays (four
species-specific qPCR assays). This can also be particularly helpful for
taxa that don't have a previously published qPCR assay, but are detected
using universal metabarcoding assays. Metabarcoding data alone only
gives compositional data, which cannot be used in a time series to
quantify environmental impacts because there is no information about
absolute eDNA concentrations. However, by anchoring or grounding
proportions using a single qPCR assay, the proportional data can be
turned into quantitative data. The species for which to run the qPCR
assay can be determined after the metabarcoding is completed; the most
commonly found species with a robust qPCR assay should be used to glean
the most information.

\hypertarget{fish-life-histories-and-expected-patterns}{%
\subsubsection{Fish life histories and expected
patterns}\label{fish-life-histories-and-expected-patterns}}

The four salmonid species in this study have different life histories
and behaviors that would impact when fish (and therefore eDNA
concentrations) occur in the creeks. Three of the four species in this
study have both freshwater and anadromous populations. Cutthroat trout
(\emph{O. clarkii}) encompasses both non-migrating, resident trout in
the creeks and coastal run cutthroat that migrate into Padden Creek from
saltwater (Bellingham Bay). Similarly, \emph{O. nerka} includes both
anadromous sockeye salmon and freshwater resident kokanee salmon and
\emph{O. mykiss} includes both anadromous steelhead trout and
non-migrating rainbow trout. Using eDNA, we cannot distinguish between
the migrating and non-migrating subspecies of \emph{O. clarkii},
\emph{O. nerka}, and \emph{O. mykiss}. Therefore, our eDNA
concentrations might reflect contributions from both migrating and
non-migrating individuals at any given time point in the dataset.

For these four anadromous salmonids, the run timings for the migrating
populations vary for each species in the study area (Bellingham, WA).
Adult coastal cutthroat (\emph{O. clarkii}) are documented to run
throughout the entire year, whereas coho salmon (\emph{O. kisutch}) run
from September to December, sockeye salmon (\emph{O. nerka}) run from
October to December, and steelhead trout (\emph{O. mykiss}) run from
November to June. For migrating coho (\emph{O. kisutch}) and steelhead
trout (\emph{O. mykiss}), juveniles may be present in the creeks
year-round (Appendix S1: Figure S4). eDNA methods at present cannot
distinguish adults versus juveniles from DNA found in a water sample.

Despite the mix of migrating and non-migrating populations and various
run timings, our metabarcoding data demonstrate that in Padden Creek,
there was a clear signal of sockeye/kokanee salmon (\emph{O. nerka})
both upstream and downstream only in November 2021 - February 2022 and
again in December 2022. This signal corresponds well with the documented
run timing of October to December and the presence of out-migrating
juveniles in early spring. In contrast, cutthroat trout (\emph{O.
clarkii}) and coho salmon (\emph{O. kisutch}) were found nearly
year-round in Padden Creek. The persistent signal from \emph{O. clarkii}
could be explained by resident cutthroat trout. However, \emph{O.
kisutch} does not have a resident subspecies and the run timing is only
documented from September to December. This could potentially be due to
juveniles maturing and residing in the creeks for 1-2 years after
hatching while adults migrate into the creeks only during the run time
to spawn. Visual surveys (e.g., snorkel surveys, electrofishing, smolt
traps) are conducted infrequently to determine adult and juvenile
salmonid abundances. Though \emph{O. kisutch} eDNA was found year round,
the highest concentrations were found near the expected run timing and
the life history of \emph{O. kisutch} includes rearing year-round in
freshwater. Finally, though the lowest concentrations on average,
rainbow/steelhead trout (\emph{O. mykiss}) was also found nearly
year-round in Padden Creek, which could be contributions from migrating
steelhead (November to June), juveniles maturing and migrating, or from
resident rainbow trout. Though the \emph{O. mykiss} signal is found
year-round, the highest concentrations do seem to correspond with the
steelhead run timing.

\hypertarget{interpreting-edna-with-respect-to-fish-abundance-and-flow}{%
\subsubsection{Interpreting eDNA with respect to fish abundance and
flow}\label{interpreting-edna-with-respect-to-fish-abundance-and-flow}}

By capturing residual eDNA from water samples, we are measuring a
different signal than counting how many fish are in the creek at each
time of sampling. We should not expect the eDNA concentration to
directly correlate to the number of fish in the creek at the time of
sampling. Shelton et al. (2019) used a paired eDNA sampling and seine
netting analysis to demonstrate that eDNA concentrations provide a
smoothed biological signal over space and time. We acknowledge this
smoothing effect and emphasize that in the context of using eDNA for
environmental impact assessments, it is preferable to use a survey
technique such as eDNA that integrates signal across a larger spatial
and temporal scale.

Many previous papers have commented on the ``ecology'' of eDNA and the
various processes that contribute to eDNA concentrations in
environmental samples (e.g., shedding rates, decay rates, transport)
(Barnes and Turner 2015). For example, higher concentrations of eDNA
could be the result of a greater number (or biomass) of fish present, or
increased shedding rates, or decreased decay. Many review papers
document the nuances of interpreting eDNA data and we recommend
reviewing them for a deeper understanding (see Andruszkiewicz Allan et
al. (2020) for a review on shedding and decay rates and Harrison et al.
(2019) for a review on transport). Other studies have also documented
the relative importance of eDNA transport in streams. Most notably,
Tillotson et al. (2018) measured eDNA at four sites with similar
discharge rates to the creeks in this study and specifically addressed
spatial and temporal resolutions, finding that eDNA concentrations
reflect short time- (and therefore length-) scales by comparing peaks in
eDNA concentrations to counts of salmon and accumulation by measuring
both upstream and downstream sites. The authors found that the sampling
site furthest downstream did not accumulate eDNA and that two
tributaries feeding into a main channel were additive (Tillotson et al.
2018). For more general models and empirical data documenting transport
distances in streams, see Wilcox et al. (2016), Jane et al. (2014),
Jerde et al. (2016), Shogren et al. (2016), and Civade et al. (2016).
Certainly eDNA concentrations can arise from different scenarios and
future work should continue to investigate how to tease apart the
nuances of relating eDNA concentrations to fish abundance.

In this study, to assess the impact of a culvert on fish passage, we
compare eDNA concentrations upstream and downstream at the same time
point in a given creek. The distance between the upstream and downstream
sampling was minimal (\textasciitilde60-300 m, average distance of
\textasciitilde150 m). Therefore, we assume that the small differences
in spatial and temporal scale between sampling locations is minimal such
that the impacts of these various processes will affect the downstream
and upstream concentrations equally. That is, in the upstream station,
some amount of eDNA is coming from upstream of that location into the
sampling station and leaving at the same time -- in the same way that
eDNA would be both entering and exiting the downstream station.
Additionally, at almost every single time point for all creeks and
species, the upstream DNA concentration is higher than the downstream
DNA concentration. Based on that alone, we do not expect that downstream
accumulation of salmonid DNA is occurring to bias our results of whether
fish can pass through these culverts.

\hypertarget{not-all-culverts-are-barriers-to-salmonids}{%
\subsubsection{Not all culverts are barriers to
salmonids}\label{not-all-culverts-are-barriers-to-salmonids}}

By measuring DNA concentrations of salmonid species above and below
culverts on a small spatial scale, we were able to determine how much of
a barrier each culvert was to fish passage. Barnes Creek was clearly a
very large barrier to fish passage as we only found salmonid DNA in
three months of the twelve months of sampling, and those three months
had very low concentration of salmonid DNA relative to the other creeks.
Within the treatment creek (Padden Creek), the SR-11 culvert did not
seem to be a large barrier, while the I-5 culvert clearly was a barrier,
demonstrated by the difference in salmonid composition and eDNA mass
flow rates over the course of sampling.

Here, we find instances where culverts designated as barriers were
likely not blocking fish passage, while others (Padden I-5 and Barnes
Creek) were barriers to fish passage. Importantly, this demonstrates
that collecting water samples for eDNA analysis might help to prioritize
restoration of culverts suspected to be barriers to salmonids and
provide a new method for post-restoration monitoring to confirm that the
barrier has been corrected and allows for fish passage. Given the large
amount of spending and effort required to replace culverts, this finding
is important and emphasizes the potential for new tools for
environmental impact assessments. We note that our sampling occurred
only over a short temporal scale and future work could monitor culverts
for longer time periods, different species, and different environmental
conditions.

\hypertarget{salmonids-can-quickly-recover-from-a-short-term-intervention-in-a-creek}{%
\subsubsection{Salmonids can quickly recover from a short-term
intervention in a
creek}\label{salmonids-can-quickly-recover-from-a-short-term-intervention-in-a-creek}}

The construction had remarkably minimal effects on salmonid DNA
concentrations. The disruption of disconnecting the creek, demolition of
the old culvert, installation of the new culvert, and the reconnecting
of the creek during both culvert replacement events showed almost no
change in the difference in eDNA concentrations between downstream and
upstream sampling sites. The differences in the control creeks between
upstream and downstream were often higher than the treatment creek. The
post-construction sampling point of the I-5 culvert replacement (only
one time point), does show that the composition of salmonid DNA after
replacement is now very similar to the two downstream stations, whereas
before construction compositions were very different (because the
culvert was a barrier). However, we lack the quantitative analysis as
the site upstream of SR-11 and downstream of I-5 had no quantifiable
cutthroat DNA. More time points would help demonstrate the effect of the
culvert replacement. Here we found that one culvert had very minimal
effect on salmonid passage while the other culvert had a large effect on
salmonid passage. We note that these findings are likely not universal
and certainly projects need to monitor comprehensively and
quantitatively in order to assess the passability of culverts and
impacts of construction.

\hypertarget{conclusion}{%
\subsection{Conclusion}\label{conclusion}}

It is notoriously difficult to quantify the environmental impact of
discrete human impacts on ecosystems and species. Surveying species and
communities by eDNA provides an opportunity for monitoring before,
during, and after impacts in a scaleable and cost-effective way. Here,
we demonstrate that monthly eDNA sampling before, during, and after an
intervention alongside control sites can quantify the environmental
impact of replacing a culvert. We found that in our treatment creek and
control sites, four of the six barriers did not prohibit salmonid
passage. We found that of the two culvert replacements in the treatment
creek, one was a barrier and one was not, but both had minimal impacts
on the four salmonid species monitored over the course of construction.
We also provide a framework in which compositional metabarcoding data
can be linked with qPCR data to obtain quantitative estimates of eDNA
concentrations of many species. This provides a practical way to utilize
the large amount of information from metabarcoding data without needing
a unique qPCR assay for every species of interest. Environmental DNA is
moving into practice and this study demonstrates how eDNA can be broadly
used for environmental impact assessments for a wide range of species
and environments.

\hypertarget{acknowledgements}{%
\subsection{Acknowledgements}\label{acknowledgements}}

This work was made possible by a grant from OceanKind to Ryan Kelly. The
funders had no role in study design, data collection and analysis,
decision to publish, or preparation of the manuscript. Figure 3 was
created with BioRender.com. We thank Tammy Schmidt and Susan Kanzler
from Washington Department of Transportation for facilitating access to
field sites and providing helpful feedback throughout the project. We
also thank Jenna McLaughlin, Joe Duprey and Ally Im for help field
sampling and Dr.~Ramon Gallego, Dr.~Kim Parsons, and the University of
Washington's Northwest Genomics Center for sequencing support.
Dr.~Braeden Van Deynze, Dr.~Sunny Jardine, and Dr.~Julian Olden provided
helpful insight into culverts and salmonid life histories. We thank
Katherine Pearson Maslenikov at the Burke Museum of Natural History and
Culture for providing voucher specimens for Sanger sequencing. Thanks to
Dr.~Jameal Samhouri and Dr.~Chris Sergeant for reviewing the manuscript.

\hypertarget{conflict-of-interest-statement}{%
\subsection{Conflict of Interest
Statement}\label{conflict-of-interest-statement}}

The authors declare there are no conflicts of interest.

\newpage

\hypertarget{figures}{%
\subsection{Figures}\label{figures}}

\begin{itemize}
\setlength{\itemindent}{4em}
\item[\textbf{Figure 1.}] Map of sampling locations near Bellingham, Washington. In the red subset, triangles designate the downstream sampling location and circles designate the upstream sampling location. Padden Creek (the treatment creek where the two culverts were replaced) is shown in the blue subset where downstream is a triangle, upstream of the first culvert (SR-11) is a circle, and upstream of the second culvert (I-5) is a square.
\item[\textbf{Figure 2.}] Discharge (m\textsuperscript{3}/s) in Padden (USGS Gauge 12201905), Chuckanut (USGS Gauge 12201700), and Squalicum (USGS Gauge 12204010) Creeks over the course of sampling. Gauges at Chuckanut and Squalicum Creek went offline in November 2021 after a major storm event. Portage Creek and Barnes Creek did not have stream gauges. Circles designate the day of sampling. For Padden Creek, the nearest 15 minute interval of flow was used. For Chuckanut and Squalicum Creeks, the correction factor from five years of historical data from Padden Creek was used (see methods section and Appendix S1: Figure S2 and Appendix S1: Figure S3).
\item[\textbf{Figure 3.}] Conceptual figure of different datasets and models used for analyses. \* indicates that here, biological replicates are different dilutions of the synthetic gBlock. \*\* indicates that for most samples, only one technical replicate was sequenced but for one sample per sampling month, three technical replicates were sequenced to check for consistency across replicates. \*\*\* indicates that here, the three biological replicates indicate three different mock communities with varying species compositions, but all containing the four salmonids of interest. Created with BioRender.com.
\item[\textbf{Figure 4.}] Compositions of salmonid DNA in control creeks as determined by metabarcoding after correction for amplification bias. Grey shading denotes time points that were not sampled (Barnes and Chuckanut Creeks after March 2023 and Squalicum Creek in September 2021 which was dry). The empty bars in the Barnes upstream sites indicate that no salmonid DNA was found at those time points.
\item[\textbf{Figure 5.}] Compositions of salmonid DNA in Padden Creek as determined by metabarcoding after correction for amplification bias. The empty bar in the March 2023 in Up5 indicates that no salmonid DNA was found. The vertical dashed lines indicates the time period in which the culverts were replaced (SR-11 and I-5, respectively).]
\item[\textbf{Figure 6.}] Absolute mass flow rate (log copies/s) of cutthroat trout (\textbf{O. clarkii}) as measured by qPCR after flow correction. Note that Barnes Creek and Chuckanut Creek were not sampled after February 2022. Red crosses show the limit of detection for each species and time point, which changes with flow rate and total volume filtered per sample.
\item[\textbf{Figure 7.}] Trends in mass flow rate (log copies/s) for each of four salmonid species across creeks and across time as estimated by eDNA analysis. Points represent posterior means for the linear mixed effects model and error bars represent the 95% posterior confidence interval. Colors indicate station upstream (black) or downstream (grey) of the culvert. Padden has an additional sampling site upstream of the second culvert (I-5; blue). Yellow shading indicates the time period in which the culverts in the treatment creek (Padden Creek) were replaced. Grey shading indicates time points that were not sampled (Barnes and Chuckanut after February 2022). Time points with no data had no sequencing reads corresponding to that species or no quantifiable cutthroat DNA by qPCR.
\item[\textbf{Figure 8.}] The effect of culvert on salmonid abundance summed across all species and creeks by time. The y-axis shows the log-fold change in eDNA mass flow rate (copies/s) between upstream and downstream, normalized by upstream mass flow rate. The box boundaries correspond to the 25th and 75th percentiles; the whiskers correspond to 1.5 times the interquartile range. Here, negative values imply that eDNA mass flow rates are higher downstream than upstream. Samples with very low eDNA mass flow rates (less than 150 copies/s) were removed before plotting to remove extreme proportional values due to large denominators. Grey stars indicate times when no samples were taken.
\item[\textbf{Figure 9.}] Log-fold change in eDNA mass flow rate between upstream and downstream over time. Size of circles corresponds to the discharge in each creek at each time point. Color of circles corresponds to each creek. Each creek and time point has up to four circles of the same color for the four salmonid species.
\item[\textbf{Figure 10.}] Effect of construction on log-fold change in eDNA mass flow rate upstream to downstream in Padden Creek. Yellow shading shows when construction started for each of the two culverts. Grey points show the corresponding log-fold changes in control creeks and black points show Padden Creek. Sockeye/kokanee salmon (\textbf{O. nerka}) was only found in Padden Creek so other creeks are not shown. Samples with very low eDNA mass flow rates (less than 150 copies/s) were removed before plotting to remove extreme proportional values due to large denominators.

\end{itemize}

\newpage

\hypertarget{references}{%
\subsection*{References}\label{references}}
\addcontentsline{toc}{subsection}{References}

\hypertarget{refs}{}
\begin{CSLReferences}{1}{0}
\leavevmode\vadjust pre{\hypertarget{ref-andruszkiewiczallan2020}{}}%
Andruszkiewicz Allan, E., W. G. Zhang, A. Lavery, and A. Govindarajan.
2020. \href{https://doi.org/10.1002/edn3.141}{Environmental DNA shedding
and decay rates from diverse animal forms and thermal regimes}.
Environmental DNA:edn3.141.

\leavevmode\vadjust pre{\hypertarget{ref-barnes2015}{}}%
Barnes, M. A., and C. R. Turner. 2015. The ecology of environmental DNA
and implications for conservation genetics. Conservation Genetics
17:117.

\leavevmode\vadjust pre{\hypertarget{ref-benedetti-cecchi2001}{}}%
Benedetti-Cecchi, L. 2001.
\href{https://doi.org/10.1890/1051-0761(2001)011\%5B0783:BBOOES\%5D2.0.CO;2}{Beyond
Baci: Optimization of Environmental Sampling Designs Through Monitoring
and Simulation}. Ecological Applications 11:783--799.

\leavevmode\vadjust pre{\hypertarget{ref-buxton2021}{}}%
Buxton, A., E. Matechou, J. Griffin, A. Diana, and R. A. Griffiths.
2021. \href{https://doi.org/10.1038/s41598-021-91166-7}{Optimising
sampling and analysis protocols in environmental DNA studies}.
Scientific Reports 11:11637.

\leavevmode\vadjust pre{\hypertarget{ref-callahan2016}{}}%
Callahan, B. J., P. J. McMurdie, M. J. Rosen, A. W. Han, A. J. A.
Johnson, and S. P. Holmes. 2016.
\href{https://doi.org/10.1038/nmeth.3869}{DADA2: High resolution sample
inference from illumina amplicon data}. Nature methods 13:581--583.

\leavevmode\vadjust pre{\hypertarget{ref-camacho2009}{}}%
Camacho, C., G. Coulouris, V. Avagyan, N. Ma, J. Papadopoulos, K.
Bealer, and T. L. Madden. 2009. BLAST+: Architecture and applications.
BMC Bioinformatics 10:4219.

\leavevmode\vadjust pre{\hypertarget{ref-cityofbellingham2015}{}}%
City of Bellingham. 2015. Urban spawner surveys.

\leavevmode\vadjust pre{\hypertarget{ref-civade2016}{}}%
Civade, R. l., T. Dejean, A. Valentini, N. Roset, J.-C. Raymond, A.
Bonin, P. Taberlet, and D. Pont. 2016. Spatial representativeness of
environmental DNA metabarcoding signal for fish biodiversity assessment
in a natural freshwater system. PLOS ONE 11:e015736619.

\leavevmode\vadjust pre{\hypertarget{ref-devargas2015}{}}%
De Vargas, C., S. Audie, N. Henry, J. Decelle, F. Mahe, R. Logares, E.
Lara, C. Berney, N. Le Bescot, I. Probert, M. Carmichael, J. Poulain, S.
Romac, S. Colin, J.-M. Aury, L. Bittner, S. Chaffron, M. Dunthorn, S.
Engelen, O. Flegontova, L. Guidi, A. Horak, O. Jaillon, G. Lima-Mendez,
J. Lukes, S. Malviya, R. Morard, M. Mulot, E. Scalco, R. Siano, F.
Vincent, A. Zingone, C. Dimier, M. Picheral, S. Searson, S.
Kandels-Lewis, T. O. coordinators, S. G. Acinas, P. Bork, C. Bowler, G.
Gorsky, N. Grimsley, P. Hingamp, D. Iudicone, F. Not, H. Ogata, S.
Pesant, J. Raes, M. E. Sieracki, S. Speich, L. Stemmann, S. Sunagawa, J.
Weissenbach, P. Wincker, and E. Karsenti. 2015. Eukaryotic plankton
diversity in the sunlit ocean. Science 348:112.

\leavevmode\vadjust pre{\hypertarget{ref-duda2021}{}}%
Duda, J. J., M. S. Hoy, D. M. Chase, G. R. Pess, S. J. Brenkman, M. M.
McHenry, and C. O. Ostberg. 2021.
\href{https://doi.org/10.1002/edn3.134}{Environmental DNA is an
effective tool to track recolonizing migratory fish following
large-scale dam removal}. Environmental DNA 3:121--141.

\leavevmode\vadjust pre{\hypertarget{ref-frankiewicz2021}{}}%
Frankiewicz, P., A. Radecki-Pawlik, A. Wałęga, M. Łapińska, and A.
Wojtal-Frankiewicz. 2021.
\href{https://doi.org/10.1139/er-2020-0126}{Small hydraulic structures,
big environmental problems: is it possible to mitigate the negative
impacts of culverts on stream biota?} Environmental Reviews 29:510--528.

\leavevmode\vadjust pre{\hypertarget{ref-gloor2016}{}}%
Gloor, G. B., J. M. Macklaim, M. Vu, and A. D. Fernandes. 2016.
\href{https://doi.org/10.17713/ajs.v45i4.122}{Compositional uncertainty
should not be ignored in high-throughput sequencing data analysis}.
Austrian Journal of Statistics 45:73--87.

\leavevmode\vadjust pre{\hypertarget{ref-gold2023}{}}%
Gold, Z., A. O. Shelton, H. R. Casendino, J. Duprey, R. Gallego, A. V.
Cise, M. Fisher, A. J. Jensen, E. D'Agnese, E. A. Allan, A. Ramón-Laca,
M. Garber-Yonts, M. Labare, K. M. Parsons, and R. P. Kelly. 2023.
\href{https://doi.org/10.1371/journal.pone.0285674}{Signal and noise in
metabarcoding data}. PLOS ONE 18:e0285674.

\leavevmode\vadjust pre{\hypertarget{ref-harrison2019}{}}%
Harrison, J. B., J. M. Sunday, and S. M. Rogers. 2019.
\href{https://doi.org/10.1098/rspb.2019.1409}{Predicting the fate of
eDNA in the environment and implications for studying biodiversity}.
Proceedings of the Royal Society B: Biological Sciences 286:20191409.

\leavevmode\vadjust pre{\hypertarget{ref-hoshino2021}{}}%
Hoshino, T., R. Nakao, H. Doi, and T. Minamoto. 2021.
\href{https://doi.org/10.1038/s41598-021-83318-6}{Simultaneous absolute
quantification and sequencing of fish environmental DNA in a mesocosm by
quantitative sequencing technique}. Scientific Reports 11:4372.

\leavevmode\vadjust pre{\hypertarget{ref-jane2014}{}}%
Jane, S. F., T. M. Wilcox, K. S. McKelvey, M. K. Young, M. K. Schwartz,
W. H. Lowe, B. H. Letcher, and A. R. Whiteley. 2014. Distance, flow and
PCR inhibition: eDNA dynamics in two headwater streams. Molecular
Ecology Resources 15:216227.

\leavevmode\vadjust pre{\hypertarget{ref-jerde2016}{}}%
Jerde, C. L., B. P. Olds, A. J. Shogren, E. A. Andruszkiewicz, A. R.
Mahon, D. Bolster, and J. L. Tank. 2016. Influence of stream bottom
substrate on retention and transport of vertebrate environmental DNA.
Environmental Science \& Technology 50:87708779.

\leavevmode\vadjust pre{\hypertarget{ref-kelly2017}{}}%
Kelly, R. P., C. J. Closek, J. L. O'Donnell, J. E. Kralj, A. O. Shelton,
and J. F. Samhouri. 2017. Genetic and manual survey methods yield
different and complementary views of an ecosystem. Frontiers in Marine
Science 3:73511.

\leavevmode\vadjust pre{\hypertarget{ref-kelly2014}{}}%
Kelly, R. P., J. A. Port, K. M. Yamahara, and L. B. Crowder. 2014. Using
environmental DNA to census marine fishes in a large mesocosm. PLOS ONE
9:e8617511.

\leavevmode\vadjust pre{\hypertarget{ref-kelly2019}{}}%
Kelly, R. P., A. O. Shelton, and R. Gallego. 2019.
\href{https://doi.org/10.1038/s41598-019-48546-x}{Understanding PCR
Processes to Draw Meaningful Conclusions from Environmental DNA
Studies}. Scientific Reports 9:12133.

\leavevmode\vadjust pre{\hypertarget{ref-klein2022}{}}%
Klein, S. G., N. R. Geraldi, A. Anton, S. Schmidt-Roach, M. Ziegler, M.
J. Cziesielski, C. Martin, N. Rädecker, T. L. Frölicher, P. J. Mumby, J.
M. Pandolfi, D. J. Suggett, C. R. Voolstra, M. Aranda, and Carlos. M.
Duarte. 2022. \href{https://doi.org/10.1111/gcb.15818}{Projecting coral
responses to intensifying marine heatwaves under ocean acidification}.
Global Change Biology 28:1753--1765.

\leavevmode\vadjust pre{\hypertarget{ref-lackey2003}{}}%
Lackey, R. 2003.
\href{https://doi.org/10.1080/16226510390856529}{Pacific Northwest
Salmon: Forecasting Their Status in 2100}. Reviews in Fisheries Science
11:35--88.

\leavevmode\vadjust pre{\hypertarget{ref-leray2013}{}}%
Leray, M., J. Y. Yang, C. P. Meyer, S. C. Mills, N. Agudelo, V. Ranwez,
J. T. Boehm, and R. J. Machida. 2013. A new versatile primer set
targeting a short fragment of the mitochondrial COI region for
metabarcoding metazoan diversity: Application for characterizing coral
reef fish gut contents. Frontiers in Zoology 10:114.

\leavevmode\vadjust pre{\hypertarget{ref-long2018}{}}%
Long, J. W., and F. K. Lake. 2018.
\href{https://www.jstor.org/stable/26799109}{Escaping social-ecological
traps through tribal stewardship on national forest lands in the pacific
northwest, united states of america}. Ecology and Society 23.

\leavevmode\vadjust pre{\hypertarget{ref-maasri2022}{}}%
Maasri, A., S. C. Jähnig, M. C. Adamescu, R. Adrian, C. Baigun, D. J.
Baird, A. Batista-Morales, N. Bonada, L. E. Brown, Q. Cai, J. V.
Campos-Silva, V. Clausnitzer, T. Contreras-MacBeath, S. J. Cooke, T.
Datry, G. Delacámara, L. De Meester, K.-D. B. Dijkstra, V. T. Do, S.
Domisch, D. Dudgeon, T. Erös, H. Freitag, J. Freyhof, J. Friedrich, M.
Friedrichs-Manthey, J. Geist, M. O. Gessner, P. Goethals, M. Gollock, C.
Gordon, H.-P. Grossart, G. Gulemvuga, P. E. Gutiérrez-Fonseca, P. Haase,
D. Hering, H. J. Hahn, C. P. Hawkins, F. He, J. Heino, V. Hermoso, Z.
Hogan, F. Hölker, J. M. Jeschke, M. Jiang, R. K. Johnson, G. Kalinkat,
B. K. Karimov, A. Kasangaki, I. A. Kimirei, B. Kohlmann, M. Kuemmerlen,
J. J. Kuiper, B. Kupilas, S. D. Langhans, R. Lansdown, F. Leese, F. S.
Magbanua, S. S. Matsuzaki, M. T. Monaghan, L. Mumladze, J. Muzon, P. A.
Mvogo Ndongo, J. C. Nejstgaard, O. Nikitina, C. Ochs, O. N. Odume, J. J.
Opperman, H. Patricio, S. U. Pauls, R. Raghavan, A. Ramírez, B. Rashni,
V. Ross-Gillespie, M. J. Samways, R. B. Schäfer, A. Schmidt-Kloiber, O.
Seehausen, D. N. Shah, S. Sharma, J. Soininen, N. Sommerwerk, J. D.
Stockwell, F. Suhling, R. D. Tachamo Shah, R. E. Tharme, J. H. Thorp, D.
Tickner, K. Tockner, J. D. Tonkin, M. Valle, J. Vitule, M. Volk, D.
Wang, C. Wolter, and S. Worischka. 2022.
\href{https://doi.org/10.1111/ele.13931}{A global agenda for advancing
freshwater biodiversity research}. Ecology Letters 25:255--263.

\leavevmode\vadjust pre{\hypertarget{ref-macpherson2012}{}}%
MacPherson, L. M., M. G. Sullivan, A. Lee Foote, and C. E. Stevens.
2012. \href{https://doi.org/10.1080/02755947.2012.686004}{Effects of
Culverts on Stream Fish Assemblages in the Alberta Foothills}. North
American Journal of Fisheries Management 32:480--490.

\leavevmode\vadjust pre{\hypertarget{ref-martin2012}{}}%
Martin, C. J. B., B. J. Allen, and C. G. Lowe. 2012.
\href{https://doi.org/10.3160/0038-3872-111.2.119}{Environmental impact
assessment: Detecting changes in fish community structure in response to
disturbance with an asymmetric multivariate BACI sampling design}.
Bulletin, Southern California Academy of Sciences 111:119--131.

\leavevmode\vadjust pre{\hypertarget{ref-martin2011}{}}%
Martin, M. 2011. \href{https://doi.org/10.14806/ej.17.1.200}{Cutadapt
removes adapter sequences from high-throughput sequencing reads}.
EMBnet.journal 17:10.

\leavevmode\vadjust pre{\hypertarget{ref-martinez2013}{}}%
Martinez, R. 2013. United states v. washington.

\leavevmode\vadjust pre{\hypertarget{ref-mccall2014}{}}%
McCall, M. N., H. R. McMurray, H. Land, and A. Almudevar. 2014.
\href{https://doi.org/10.1093/bioinformatics/btu239}{On non-detects in
qPCR data}. Bioinformatics 30:2310--2316.

\leavevmode\vadjust pre{\hypertarget{ref-mclaren2019}{}}%
McLaren, M. R., A. D. Willis, and B. J. Callahan. 2019.
\href{https://doi.org/10.7554/eLife.46923}{Consistent and correctable
bias in metagenomic sequencing experiments}. eLife 8:e46923.

\leavevmode\vadjust pre{\hypertarget{ref-morgan2012}{}}%
Morgan, R. K. 2012.
\href{https://doi.org/10.1080/14615517.2012.661557}{Environmental impact
assessment: The state of the art}. Impact Assessment and Project
Appraisal 30:5--14.

\leavevmode\vadjust pre{\hypertarget{ref-moss2022}{}}%
Moss, W. E., L. R. Harper, M. A. Davis, C. S. Goldberg, M. M. Smith, and
P. T. J. Johnson. 2022.
\href{https://doi.org/10.1002/ecs2.3941}{Navigating the trade-offs
between environmental DNA and conventional field surveys for improved
amphibian monitoring}. Ecosphere 13:e3941.

\leavevmode\vadjust pre{\hypertarget{ref-muha2017}{}}%
Muha, T. P., M. Rodríguez-Rey, M. Rolla, and E. Tricarico. 2017. Using
environmental DNA to improve species distribution models for freshwater
invaders. Frontiers in Ecology and Evolution 5:143957.

\leavevmode\vadjust pre{\hypertarget{ref-nathan2018}{}}%
Nathan, L. R., A. A. Smith, A. B. Welsh, and J. C. Vokoun. 2018.
\href{https://doi.org/10.1016/j.ecolind.2017.08.033}{Are culvert
assessment scores an indicator of Brook Trout Salvelinus fontinalis
population fragmentation?} Ecological Indicators 84:208--217.

\leavevmode\vadjust pre{\hypertarget{ref-ogram1987}{}}%
Ogram, A., G. Sayler, and T. Barkay. 1987. The extraction and
purification of microbial DNA from sediments. Journal of Microbiological
Methods 7:5766.

\leavevmode\vadjust pre{\hypertarget{ref-ogren2015}{}}%
Ogren, S. A., and C. J. Huckins. 2015.
\href{https://doi.org/10.1111/rec.12250}{Culvert replacements:
improvement of stream biotic integrity?} Restoration Ecology
23:821--828.

\leavevmode\vadjust pre{\hypertarget{ref-pont2022}{}}%
Pont, D., P. Meulenbroek, V. Bammer, T. Dejean, T. Erős, P. Jean, M.
Lenhardt, C. Nagel, L. Pekarik, M. Schabuss, B. C. Stoeckle, E. Stoica,
H. Zornig, A. Weigand, and A. Valentini. 2022.
\href{https://doi.org/10.1111/1755-0998.13715}{Quantitative monitoring
of diverse fish communities on a large scale combining eDNA
metabarcoding and qPCR}. Molecular Ecology Resources n/a.

\leavevmode\vadjust pre{\hypertarget{ref-port2015}{}}%
Port, J. A., J. L. O'Donnell, O. C. Romero-Maraccini, P. R. Leary, S. Y.
Litvin, K. J. Nickols, K. M. Yamahara, and R. P. Kelly. 2015. Assessing
vertebrate biodiversity in a kelp forest ecosystem using environmental
DNA. Molecular Ecology 25:527541.

\leavevmode\vadjust pre{\hypertarget{ref-price2010}{}}%
Price, D. M., T. Quinn, and R. J. Barnard. 2010.
\href{https://doi.org/10.1577/M10-004.1}{Fish Passage Effectiveness of
Recently Constructed Road Crossing Culverts in the Puget Sound Region of
Washington State}. North American Journal of Fisheries Management
30:1110--1125.

\leavevmode\vadjust pre{\hypertarget{ref-rcoreteam2017}{}}%
R Core Team. 2017. R: A language and environment for statistical
computing. R Foundation for Statistical Computing.

\leavevmode\vadjust pre{\hypertarget{ref-rondon2000}{}}%
Rondon, M. R., P. R. August, A. D. Bettermann, S. F. Brady, T. H.
Grossman, M. R. Liles, K. A. Loiacono, B. A. Lynch, I. A. MacNeil, C.
Minor, C. L. Tiong, M. Gilman, M. S. Osburne, J. Clardy, J. Handelsman,
and R. M. Goodman. 2000.
\href{https://doi.org/10.1128/AEM.66.6.2541-2547.2000}{Cloning the soil
metagenome: a strategy for accessing the genetic and functional
diversity of uncultured microorganisms}. Applied and Environmental
Microbiology 66:2541--2547.

\leavevmode\vadjust pre{\hypertarget{ref-rubin2017}{}}%
Rubin, Z., G. M. Kondolf, and B. Rios-Touma. 2017.
\href{https://doi.org/10.3390/w9030174}{Evaluating Stream Restoration
Projects: What Do We Learn from Monitoring?} Water 9:174.

\leavevmode\vadjust pre{\hypertarget{ref-ruppert2019}{}}%
Ruppert, K. M., R. J. Kline, and M. S. Rahman. 2019.
\href{https://doi.org/10.1016/j.gecco.2019.e00547}{Past, present, and
future perspectives of environmental DNA (eDNA) metabarcoding: A
systematic review in methods, monitoring, and applications of global
eDNA}. Global Ecology and Conservation 17:e00547.

\leavevmode\vadjust pre{\hypertarget{ref-schmidhauser1976a}{}}%
Schmidhauser, J. R. 1976. Struggles for cultural survival: The fishing
rights of the treaty tribes of the Pacific Northwest. Notre Dame Law
52:30--40.

\leavevmode\vadjust pre{\hypertarget{ref-seymour2021}{}}%
Seymour, M., F. K. Edwards, B. J. Cosby, I. Bista, P. M. Scarlett, F. L.
Brailsford, H. C. Glanville, M. de Bruyn, G. R. Carvalho, and S. Creer.
2021. \href{https://doi.org/10.1038/s42003-021-02031-2}{Environmental
DNA provides higher resolution assessment of riverine biodiversity and
ecosystem function via spatio-temporal nestedness and turnover
partitioning}. Communications Biology 4:1--12.

\leavevmode\vadjust pre{\hypertarget{ref-shelton}{}}%
Shelton, A. O., Z. J. Gold, A. J. Jensen, E. D'Agnese, E. Andruszkiewicz
Allan, A. Van Cise, R. Gallego, A. Ramón-Laca, M. Garber-Yonts, K.
Parsons, and R. P. Kelly. 2022.
\href{https://doi.org/10.1002/ecy.3906}{Toward quantitative
metabarcoding}. Ecology n/a:e3906.

\leavevmode\vadjust pre{\hypertarget{ref-shelton2019}{}}%
Shelton, A. O., R. P. Kelly, J. L. O'Donnell, L. Park, P. Schwenke, C.
Greene, R. A. Henderson, and E. M. Beamer. 2019.
\href{https://doi.org/10.1016/j.biocon.2019.07.003}{Environmental DNA
provides quantitative estimates of a threatened salmon species}.
Biological Conservation 237:383--391.

\leavevmode\vadjust pre{\hypertarget{ref-shelton2016}{}}%
Shelton, A. O., J. L. O'Donnell, J. F. Samhouri, N. C. Lowell, G. D.
Williams, and R. P. Kelly. 2016. A framework for inferring biological
communities from environmental DNA:115.

\leavevmode\vadjust pre{\hypertarget{ref-shogren2016}{}}%
Shogren, A. J., J. L. Tank, E. A. Andruszkiewicz, B. P. Olds, C. L.
Jerde, and D. Bolster. 2016. Modelling the transport of environmental
DNA through a porous substrate using continuous flow-through column
experiments. Journal of The Royal Society Interface 13:2016029011.

\leavevmode\vadjust pre{\hypertarget{ref-silverman2021}{}}%
Silverman, J. D., R. J. Bloom, S. Jiang, H. K. Durand, E. Dallow, S.
Mukherjee, and L. A. David. 2021.
\href{https://doi.org/10.1371/journal.pcbi.1009113}{Measuring and
mitigating PCR bias in microbiota datasets}. PLoS Computational Biology
17:e1009113.

\leavevmode\vadjust pre{\hypertarget{ref-standevelopmentteam2022}{}}%
Stan Development Team. 2022. \href{https://mc-stan.org/}{RStan: The r
interface to stan.}

\leavevmode\vadjust pre{\hypertarget{ref-stat2017}{}}%
Stat, M., M. J. Huggett, R. Bernasconi, J. D. DiBattista, T. E. Berry,
S. J. Newman, E. S. Harvey, and M. Bunce. 2017. Ecosystem biomonitoring
with eDNA: Metabarcoding across the tree of life in a tropical marine
environment. Scientific Reports:111.

\leavevmode\vadjust pre{\hypertarget{ref-taberlet2012}{}}%
Taberlet, P., E. Coissac, M. Hajibabaei, and L. H. Rieseberg. 2012.
Environmental DNA. Molecular Ecology 21:17891793.

\leavevmode\vadjust pre{\hypertarget{ref-thalinger2019}{}}%
Thalinger, B., E. Wolf, M. Traugott, and J. Wanzenböck. 2019.
\href{https://doi.org/10.1038/s41598-019-51398-0}{Monitoring spawning
migrations of potamodromous fish species via eDNA}. Scientific Reports
9:15388.

\leavevmode\vadjust pre{\hypertarget{ref-thomas2018}{}}%
Thomas, A. C., J. Howard, P. L. Nguyen, T. A. Seimon, and C. S.
Goldberg. 2018. ANDe {\texttrademark}: A fully integrated environmental
DNA sampling system. Methods in Ecology and Evolution 9:13791385.

\leavevmode\vadjust pre{\hypertarget{ref-thomas2019}{}}%
Thomas, A. C., P. L. Nguyen, J. Howard, and C. S. Goldberg. 2019.
\href{https://doi.org/10.1111/2041-210X.13212}{A self-preserving,
partially biodegradable eDNA filter}. Methods in Ecology and Evolution
10:1136--1141.

\leavevmode\vadjust pre{\hypertarget{ref-thomsen2015}{}}%
Thomsen, P. F., and E. Willerslev. 2015. Environmental DNA: An emerging
tool in conservation for monitoring past and present biodiversity.
Biological Conservation 183:418.

\leavevmode\vadjust pre{\hypertarget{ref-tillotson2018}{}}%
Tillotson, M. D., R. P. Kelly, J. J. Duda, M. Hoy, J. Kralj, and T. P.
Quinn. 2018. Concentrations of environmental DNA (eDNA) reflect spawning
salmon abundance at fine spatial and temporal scales. Biological
Conservation 220:111.

\leavevmode\vadjust pre{\hypertarget{ref-turnbaugh2007}{}}%
Turnbaugh, P. J., R. E. Ley, M. Hamady, C. M. Fraser-Liggett, R. Knight,
and J. I. Gordon. 2007. \href{https://doi.org/10.1038/nature06244}{The
Human Microbiome Project}. Nature 449:804--810.

\leavevmode\vadjust pre{\hypertarget{ref-u.s.geologicalsurvey1994}{}}%
U. S. Geological Survey. 1994.
\href{https://doi.org/10.5066/F7P55KJN}{USGS water data for the nation}.
Retrieved from https://waterdata.usgs.gov/nwis
\textless11/30/2022\textgreater.

\leavevmode\vadjust pre{\hypertarget{ref-underwood1992}{}}%
Underwood, A. J. 1992.
\href{https://doi.org/10.1016/0022-0981(92)90094-Q}{Beyond BACI: the
detection of environmental impacts on populations in the real, but
variable, world}. Journal of Experimental Marine Biology and Ecology
161:145--178.

\leavevmode\vadjust pre{\hypertarget{ref-underwood1994}{}}%
Underwood, A. J. 1994. \href{https://doi.org/10.2307/1942110}{On Beyond
BACI: Sampling Designs that Might Reliably Detect Environmental
Disturbances}. Ecological Applications 4:3--15.

\leavevmode\vadjust pre{\hypertarget{ref-valentini2016}{}}%
Valentini, A., P. Taberlet, C. Miaud, R. l. Civade, J. E. Herder, P. F.
Thomsen, E. Bellemain, A. Besnard, E. Coissac, F. Boyer, C. Gaboriaud,
P. Jean, N. Poulet, N. Roset, G. H. Copp, P. Geniez, D. Pont, C.
Argillier, J.-M. Baudoin, T. Peroux, A. J. Crivelli, A. Olivier, M.
Acqueberge, M. Le Brun, P. R. Moller, E. Willerslev, and T. Dejean.
2016. Next-generation monitoring of aquatic biodiversity using
environmental DNA metabarcoding. Molecular Ecology 25:929942.

\leavevmode\vadjust pre{\hypertarget{ref-washingtondepartmentoffishandwildlife2019}{}}%
Washington Department of Fish and Wildlife. 2019. Fish passage
inventory, assessment, and prioritization manual.

\leavevmode\vadjust pre{\hypertarget{ref-wellman2000}{}}%
Wellman, J., D. Combs, and S. B. Cook. 2000.
\href{https://www.tandfonline.com/doi/epdf/10.1080/02705060.2000.9663750?needAccess=true\&role=button}{Long-Term
Impacts of Bridge and Culvert Construction or Replacement on Fish
Communities and Sediment Characteristics of Streams}. Journal of
Freshwater Ecology 15:317--328.

\leavevmode\vadjust pre{\hypertarget{ref-wilcox2016}{}}%
Wilcox, T. M., K. S. McKelvey, M. K. Young, A. J. Sepulveda, B. B.
Shepard, S. F. Jane, A. R. Whiteley, W. H. Lowe, and M. K. Schwartz.
2016. Understanding environmental DNA detection probabilities: A case
study using a stream-dwelling char salvelinus fontinalis. Biological
Conservation 194:209216.

\leavevmode\vadjust pre{\hypertarget{ref-wilkinson2018}{}}%
Wilkinson, S. P., S. K. Davy, M. Bunce, and M. Stat. 2018.
\href{https://doi.org/10.7287/peerj.preprints.26812v1}{Taxonomic
identification of environmental DNA with informatic sequence
classification trees.}

\leavevmode\vadjust pre{\hypertarget{ref-wood2018}{}}%
Wood, D. M., A. B. Welsh, and J. Todd Petty. 2018.
\href{https://doi.org/10.1002/nafm.10185}{Genetic Assignment of Brook
Trout Reveals Rapid Success of Culvert Restoration in Headwater
Streams}. North American Journal of Fisheries Management 38:991--1003.

\leavevmode\vadjust pre{\hypertarget{ref-yamanaka2016}{}}%
Yamanaka, H., and T. Minamoto. 2016. The use of environmental DNA of
fishes as an efficient method of determining habitat connectivity.
Ecological Indicators 62:147153.

\end{CSLReferences}

\end{document}
